\documentclass{beamer}

\mode<presentation> {
\usetheme{Berlin}
% \usecolortheme{beaver}
% \usecolortheme{crane}
% \usecolortheme{dolphin}
% \usecolortheme{lily}
% \usecolortheme{whale}
% \usecolortheme{wolverine}
}

\usefonttheme{professionalfonts}

\usepackage{graphicx}
\usepackage{booktabs}
\usepackage[UTF8,noindent]{ctexcap}
\usepackage[bookmarks=true]{hyperref}

\title[好题分享]{好题分享}

\author{Milmon}
\institute[Hailiang Junior High School]
{
Hailiang Junior High School
}
% \date{\today}
\date{2024 年 8 月 12 日}

\begin{document}

\begin{frame}
\titlepage
\end{frame}

\section{A}

\begin{frame}
\frametitle{Erdős-Ginzburg-Ziv\footnote{ETH Zurich Competitive Programming Contest Spring 2024 Problem E}}
给定 $p - 1$ 个在 $[1, p - 1]$ 范围内的数,你需要构造一棵 $p$ 个结点的树,满足:
\begin{itemize}
\item $p - 1$ 条边的边权正好是给定的 $p - 1$ 个数;
\item 令结点从 $0$ 开始编号,则对于任意 $1 \leq i < p$,结点 $0$ 到结点 $i$ 在树上的距离模 $p$ 正好等于 $i$。
\end{itemize}
$p$ 是素数,$p \leq 10^6$。
\end{frame}

\begin{frame}
\frametitle{Erdős-Ginzburg-Ziv 题解}
考虑每次找到能加的边就直接添加。\\
\pause
证明:设当前边权为 $w$,考虑链 $0 \to w \to 2w \to \cdots \to 0$(模 $p$ 意义下),由于 $p$ 是素数,所以上述链遍历所有点,那么一定存在相邻的两项前者在树上,并且后者不在树上。\\
\pause
这样我们可以得到一个 $\Theta(n^2)$ 的做法:遇到一个边权就枚举所有已经与 $0$ 号点连通的点判断是否能加入这条边。
\end{frame}

\begin{frame}
\frametitle{Erdős-Ginzburg-Ziv 题解}
考虑优化,设当前边权为 $w$,同样考虑链 $0 \to w \to 2w \to \cdots \to 0$(模 $p$ 意义下),链上一定存在一个边合法。\\
\pause
首先任意选一个不在树上的点,找到其在环上的位置,然后以 $0$ 为左端点,该位置为右端点二分,如果中点在树上就移动左端点否则移动右端点,这样左端点永远在树上,右端点永远不在树上,最终可以找到一条可以添加的边。\\
\pause
时间复杂度 $\Theta(n \log n)$。
\end{frame}

\section{B}

\begin{frame}
\frametitle{情书}
给定两个字符串 $a, b$,长度分别为 $n, m$,再给定整数 $t$,设 $s$ 表示字符串 $a$ 循环 $t$ 次得到的字符串。称 $b$ 循环任意次都是好的,求 $s$ 中有多少个子串是好的。\\
$1 \leq n, m \leq 5 \times 10^5$,$1 \leq t \leq 10^9$。
\end{frame}

\begin{frame}
\frametitle{情书题解}
考虑朴素的 dp,设在 $s$ 串中 $f(i)$ 表示以 $i$ 为结尾最多能匹配多少个 $b$,那么所有 $f$ 值的和就是答案。\\
\pause
注意到如果 $a$ 复制的若干次的 $f$ 值不受到前两个 $a$ 的影响,那么可以直接对一段 $a$ 求出贡献,然后乘上 $t$。\\
\pause
可能受影响的情况等价于把 $a$ 首尾相接形成 $b$ 复制若干次的结构。\\
\pause
判断是否是这种情况,如果是,那么 $s$ 一定形如 $Xbb\cdots bY$,其中 $X,Y$ 是零散串。直接枚举开头,求出最多向后匹配多少个 $b$ 并求组合数和即可。\\
\pause
时间复杂度 $\Theta(n\log n)$。(不知道可不可以 $\Theta(n)$,但不重要。)
\end{frame}

\section{C}

\begin{frame}
\frametitle{联通块}
给定一棵 $n$ 个结点的树,对于每一种把树划分成若干个连通块的方案,设其每个连通块的大小分别是 $a_1, a_2, \cdots, a_m$,则这个划分的权值是 $\prod\limits_{i=1}^{m} a_i^k$。其中 $k$ 是给定的正整数常数。请求出所有划分的权值的和模 $998244353$。\\
$1 \leq n, k \leq 5 \times 10^5$,$1 \leq n \times k \leq 5 \times 10^5$。
\end{frame}

\begin{frame}
\frametitle{联通块题解}
考虑动态规划。\\
\pause
记 $f(i,j)$ 表示以 $i$ 为根的子树中 $i$ 所在连通块的大小为 $j$ 时,子树内所有划分的权值之和。\\
\pause
转移枚举两个部分所选择的连通块大小。\\
\pause
时间复杂度 $\Theta(n^2)$ 同树形背包。
\end{frame}

\begin{frame}
\frametitle{联通块题解}
继续考虑动态规划。\\
\pause
记 $f(i,j)$ 表示所有 $i$ 为根的子树的划分方案中,$i$ 所在连通块的大小的 $j$ 次方乘上其他连通块的大小的 $k$ 次方的总和。\\
\pause
考虑组合意义:$k$ 次方相当于在连通块中选取 $k$ 次任意点。\\
\pause
转移时枚举在一个部分中选择的点的数量,系数是一个组合数。\\
\pause
总时间复杂度 $\Theta(n k^2)$。
\end{frame}

\begin{frame}
\frametitle{联通块题解}
一个方案是,当 $n \leq 5000$ 时选择第一种 dp,当 $n > 5000$ 时 $k \leq 100$,选择第二种 dp。\\
\pause
也可以将第二种 dp 的转移使用 NTT 优化,时间复杂度变为 $\Theta(nk \log k)$。
\end{frame}

\end{document} 
